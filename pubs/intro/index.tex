\documentclass[nonacm,timestamp,manuscript]{acmart}
\usepackage{amsmath}
\usepackage{semantic}
\usepackage{minted}
\usepackage{mathpartir}
\usepackage{stmaryrd}
\usepackage{amssymb}
\usepackage{todonotes}
\usepackage{chngcntr}
\usepackage{subcaption}
\setmonofont[Scale=0.8]{DejaVu Sans Mono}

\newcommand{\alt}{\ |\ }
\newcommand{\atom}[1]{\texttt{#1}}
\newcommand{\tuple}[1][T]{\left\{ {#1}_1, \ldots, {#1}_n \right\}}
\newcommand{\singleton}[2]{\langle #1, #2 \rangle} % chktex 1
\newcommand{\singletonex}{\singleton{T}{v}}
\newcommand{\type}{\text{T}}
\newcommand{\types}{\text{S}}
\newcommand{\atomtype}{{\type}^{*}}
\newcommand{\postype}{{\type}^{+}}
\newcommand{\postypes}{{\types}^{+}}
\newcommand{\posstype}{{\type}^{+\text{S}}}
\newcommand{\posstypes}{{\types}^{+\text{S}}}
\newcommand{\posctype}{{\type}^{+\text{C}}}
\newcommand{\negtype}{{\type}^{-}}
\newcommand{\res}[1]{\texttt{#1}}
\newcommand{\If}[3]{\res{if}\ #1\ \res{then}\ #2\ \res{else}\ #3}

\counterwithin*{equation}{section}
\counterwithin*{equation}{subsection}

\title{Typing Erlang Clauses with Flow Typing}
\author{Joseph Harrison}
\affiliation{%
  \department{School of Computing}
  \institution{University of Kent}
  \city{Canterbury}
  \country{United Kingdom}}
\email{jrh53@kent.ac.uk}

\begin{document}

\maketitle

\tableofcontents

\section{Types}

\begin{align*}
  \type ::={}&
               \atom{any} \alt
               \atom{atom} \alt \atom{boolean} \alt
               \atom{number} \alt \atom{integer} \alt
               \atom{tuple} \alt \tuple \alt
               \singletonex \alt
               \neg \type \alt
               {\type}_1 \lor {\type}_2 \alt {\type}_1 \land {\type}_2
\end{align*}

First, we have some types which are analogous to their Erlang cousins:
\atom{any}, \atom{atom}, \atom{boolean} (the atom values \texttt{true} and
\texttt{false}), \atom{number}, and \atom{integer}. There are also two types for
tuples: \atom{tuple} for tuples of an unknown size, and $\tuple$ for those with
one which is known. Those familiar with Erlang's type system will know of
singleton types, even if they are not aware of the name: a concrete value used
as a type. Here, $\singletonex$ represents a singleton of type $T$ with value
$v$.

The type $\neg \type$ represents the negation of a type. The union of (or choice
between) two types is noted as ${\type}_1 \lor {\type}_2$; this is seen in
Erlang as a vertical bar, as in ${\type}_1 | {\type}_2$. Finally, we have the
intersection of two types, written ${\type}_1 \land {\type}_2$.


\section{Language}

\subsection{Guards}

\begin{align*}
  g &{}::= \res{true} \alt \res{false} \alt \res{not}\ g \alt
      \res{if}\ g\ \res{then}\ g_t\ \res{else}\ g_f \alt
      v\ \res{is}\ T \alt v\ \res{=:=}\ l
\end{align*}

Our guards represent a subset of those in Erlang, and all of them evaluate to
either \res{true} or \res{false}, which represent success and failure,
respectively. The value of a guard is negated via the \res{not} keyword. The
binary boolean operators \res{and}/\res{andalso} do not exist; they are
implemented via the \res{if}/\res{then}/\res{else} construct. We can also check
whether a variable $v$ has type $T$ via the type check $v\ \res{is}\ T$,
returning. Finally, we can check that a variable $v$ is equal to an Erlang
literal $l$ via $v\ \res{=:=}\ l$.

\begin{definition}[Guard Evaluation]

\newcommand{\Gden}[2][\rho]{gval(#2,#1)} % chktex 36

Guards are interpreted via the $\Gden{\cdot}$ function in an
environment $\rho : v \rightharpoonup l$ which is a partial map from variable names to literal
values.



\begin{align*}
  \Gden{\res{true}
  } ={}&
         true
  && \textsc{GvalTrue} \\
  \Gden{\res{false}
  } ={}&
         false
  && \textsc{GvalFalse} \\
  \Gden{\res{if}\ g\ \res{then}\ g_t\ \res{else}\ g_f
  } ={}&
         \begin{cases}
           \Gden{g_t} & \text{if}\ \Gden{g} = true\\
           \Gden{g_f} & \text{if}\ \Gden{g} = false
         \end{cases}
  && \textsc{GvalIf} \\
  \Gden{v\ \res{is}\ T
  } ={}&
         \begin{cases}
           true & \text{if}\ v \in dom(\rho)\ \text{and}\ \rho(v) : S\ \text{and}\ T \le S \\
           false & \text{otherwise}
         \end{cases}
  && \textsc{GvalIs} \\
  \Gden{v\ \res{=:=}\ l
  } ={}&
         \begin{cases}
           true & \text{if}\ v \in dom(\rho)\ \text{and}\ \rho(v) =:= l \\
           false & otherwise
         \end{cases}
  && \textsc{GvalEq}          
\end{align*}
\end{definition}

The evaluation of a guard results in a logical $true$ or $false$ value, meaning
success or failure respectively. \textsc{GTrue} and \textsc{GFalse} simply
evaluate to their logical values, and \textsc{GIf} also behaves typically.
The \textsc{GIs} rule performs type checks on a variable: when $v$ is in the
domain of $\rho$, this returns $true$ if the type of $\rho(v)$ is a supertype of $T$.
Recall from rule \textsc{S-Refl} that the sub-typing relation is reflexive, and
that \textsc{GIs} therefore also evaluates to $true$ when the type of $\rho(v)$ is
identical to $T$. The equality check in \textsc{GEq} returns true when $v$ is
in the domain of $\rho$, and its value is equal to the literal $l$ via Erlang's
$=:=$ operator.

\subsection{Generating Type Environments}

\newcommand{\Gden}[1]{{G \llbracket{} #1 \rrbracket{}}}
\newcommand{\Proj}[2]{{\pi}_{#1} (#2)}
\newcommand{\ProjL}[1]{\Proj{1}{#1}}
\newcommand{\ProjR}[1]{\Proj{2}{#1}}
\newcommand{\JoinG}[2]{#1 \Join{} #2}
\newcommand{\Map}[2]{[#1 \mapsto{} #2]}

\begin{definition}[Environment]
  An environment is a total function $\Gamma$ which maps variables to types. The
  shorthand $\Map{x}{T}$ maps the variable $x$ to type $T$, leaving all other
  variables mapped to $\atom{any}$. It can thus be written:

  \begin{align*}
    \Map{x}{T} = \lambda y . \text{if}\ x = y\ \text{then}\ T\ \text{else}\ \atom{any}
  \end{align*}
\end{definition}

\begin{definition}[Joining Environment Sets]
  Two sets environments $a$ and $b$ can be joined via the $\Join$ operator, which
  merges each environment in $a$ with each environment in $b$ via conjunction:
  \begin{align*}
    \JoinG{a}{b} = \left\{ \lambda x. f(x) \land g(x)\ \middle|\ f \in a \land g \in b \right\}
  \end{align*}
\end{definition}

\begin{definition}[Environment Generation]

  We generate an ordered pair of \textit{sets} of type environments for a guard
  via the $\Gden{\cdot}$ function. The first element of the pair represents the
  type environments for which the guard succeeds, and the second for failure.
  \begin{align*}
    \Gden{\res{true}
    } ={}&
           \left(
           \left\{ \lambda x . \atom{any} \right\}
           ,
           \varnothing
           \right)
    && \textsc{GTrue} \\
    \Gden{\res{false}
    } ={}&
           \left(
           \varnothing
           ,
           \left\{ \lambda x . \atom{any} \right\}
           \right)
    && \textsc{GFalse} \\
    \Gden{\res{not}\ g
    } ={}&
           \left(
           \ProjR{\Gden{g}}
           ,
           \ProjL{\Gden{g}}
           \right)
    && \textsc{GNot} \\
    \Gden{\res{if}\ g\ \res{then}\ g_t\ \res{else}\ g_f
    } ={}&
           ( (\JoinG{\ProjL{\Gden{g}}}{\ProjL{\Gden{g_t}}}) \cup (\JoinG{\ProjR{\Gden{g}}}{\ProjL{\Gden{g_f}}}),
    && \textsc{GIf} \\
         & (\JoinG{\ProjL{\Gden{g}}}{\ProjR{\Gden{g_f}}} ) \cup (\JoinG{\ProjR{\Gden{g}}}{\ProjR{\Gden{g_f}}} ) ) \\
    \Gden{v\ \res{is}\ T
    } ={}&
           \left(
           \left\{ \Map{v}{T} \right\}
           ,
           \left\{ \Map{v}{\neg T} \right\}
           \right)
    && \textsc{GIs} \\
    \Gden{v\ \res{=:=}\ l
    } ={}&
           \left(
           \left\{ \Map{v}{\singleton{type(l)}{l}} \right\}
           ,
           \left\{ \Map{v}{\neg \singleton{type(l)}{l}} \right\}
           \right)
    && \textsc{GEq}
  \end{align*}
\end{definition}

The rule \textsc{GTrue} states that the guard $\res{true}$ always succeeds and
never fails. Similarly, \textsc{GFalse} never succeeds and always fails. Both
of these guards contain no type assertions, so their success and failure
environments respectively map all variables to $\atom{any}$. Negation of a
guard via \textsc{GNot} flips the pair from the guard it wraps: it succeeds in
all cases the sub-guard fails, and vice versa.

Conditional statements are the most complex, as multiple control flow paths can
be taken. Each of $g_t$ and $g_f$ may succeed or fail, but the selection of
these depends on the success or failure of $g$ respectively. Ultimately, the
guard succeeds whenever $g_t$ or $g_f$ does, and fails similarly. We must
therefore consider four situations: when $g$ succeeds, $g_t$ either succeeds or
fails, and when $g$ fails, $g_f$ either succeeds or fails. In each case, we
combine the results of the control flows, taking the join ($\Join$) of the
corresponding branches. We do not follow the definition of~\cite{Pearce2013} for
$\res{if}$ statements which are both restrictive and create an
over-approximation of the typing environment.

Type assertions (\textsc{GIs}) and equality checks (\textsc{GEq}) are much more
straightforward: they succeed when the test passes, and fail when it does not.

\begin{example}[Idempotency of $\res{if}$]
    \begin{align*}
    \Gden{\If{w\ \res{is}\ W}{\res{true}}{\res{false}}}
    =& ( \JoinG{\ProjL{\Gden{w\ \res{is}\ W}}}{\ProjL{\Gden{\res{true}}}}
       \cup \JoinG{\ProjR{\Gden{w\ \res{is}\ W}}}{\ProjL{\Gden{\res{false}}}}
       , \\
     & \JoinG{\ProjL{\Gden{w\ \res{is}\ W}}}{\ProjR{\Gden{\res{true}}}}
       \cup \JoinG{\ProjR{\Gden{w\ \res{is}\ W}}}{\ProjR{\Gden{\res{false}}}} ) \\
    =& ( \JoinG{\left\{ \Map{w}{W} \right\}}{ \left\{ \lambda x . \atom{any} \right\}}
       \cup \JoinG{\left\{ \Map{w}{\neg W} \right\}}{ \varnothing }
       , \\
     & \JoinG{\left\{ \Map{w}{W} \right\}}{\varnothing}
       \cup \JoinG{ \left\{ \Map{w}{\neg W} \right\} }{ \left\{ \lambda x . \atom{any} \right\} }
       ) \\
    =& ( \left\{ \Map{w}{W \land{\atom{any}} \right\} } \cup \varnothing
       ,
       \varnothing \cup \left\{ \Map{w}{\neg W \land \atom{any}} \right\}
       ) \\
    =& ( \Map{w}{W}, \Map{w}{\neg W} )
  \end{align*}

\todo[inline]{this is also an example of idempotency of abstraction of $g$ to
$\If{g}{\res{true}}{\res{false}}$}
\end{example}

\begin{proof}[Idempotency of $\res{if}$]
  \begin{align*}
    \forall g . \Gden{\If{g}{\res{true}}{\res{false}}} = \Gden{g}
  \end{align*}
  Trivial by induction on $g$. \todo{is it?}
\end{proof}

\begin{example}[Guard interpretation]
  \begin{align*}
    \Gden{\If{v\ \res{is}\ S}{( \If{w\ \res{is}\ W}{\res{true}}{\res{false}} )}{v\ \res{is}\ T}}
  \end{align*}
  This can be simplified via idempotency of $\res{if}$ to
  \begin{align*}
    \Gden{\If{v\ \res{is}\ S}{w\ \res{is}\ W}{v\ \res{is}\ T}}
  \end{align*}

  There are four possibilities when evaluating this guard, depending on whether
  $v\ \res{is}\ S$ succeeds. If the test succeeds, then $w\ \res{is}\ W$ either
  fails or succeeds, and if the test fails, then $v\ \res{is}\ T$ either
  succeeds or fails. There are two control paths in which the guard succeeds
  (success of the test and success of the true branch, or failure of the test
  and success of the false branch), and two in which it fails (success of the
  test and failure of the true branch, or failure of the test and failure of the
  true branch). Therefore, we expect four type environments: two representing
  success, and two for failure.
  
  \begin{align*}
    \Gden{\If{v\ \res{is}\ S}{w\ \res{is}\ W}{v\ \res{is}\ T}}
    =&
       ( \JoinG{\left\{ \Map{v}{S} \right\}}{\left\{ \Map{w}{W} \right\}}
       \cup \JoinG{\left\{ \Map{v}{\neg S} \right\}}{\left\{ \Map{v}{T} \right\} }
       , \\
     & \JoinG{\left\{ \Map{v}{S} \right\}}{\left\{ \Map{w}{\neg W} \right\}}
       \cup \JoinG{\left\{ \Map{v}{\neg S} \right\}}{\left\{ \Map{v}{\neg T} \right\}}
       ) \\
    =&
       ( \left\{ [ v \mapsto S, w \mapsto W ] \right\}
       \cup \left\{ [ v \mapsto \neg S \land T ] \right\}
       , \\
     & \left\{ [ v \mapsto S, w \mapsto \neg W ] \right\}
       \cup \left\{ [ v \mapsto \neg S \land \neg T ] \right\}
       ) \\
    =& ( \left\{ [ v \mapsto S, w \mapsto W ], [ v \mapsto \neg S \land T ] \right\}
       , \\
     &  \left\{ [ v \mapsto S, w \mapsto \neg W ], [ v \mapsto \neg S \land \neg T ] \right\} )
  \end{align*}
  \todo[inline]{This seems better than Pearce's over-approximation which from
    my understanding would give a single type environment
    $[ v \mapsto S \lor T, w \mapsto W \lor \neg W ]$. That allows for $v : S$ and $w : \neg W$,
    which is \textit{not} a successful control path.
  }
\end{example}





\subsection{Patterns}
\begin{align*}
  p &{}::= v \alt l \alt \{ p_1, \ldots, p_n \} \alt p = v
\end{align*}

Patterns may be simply free variables ($v$), a tuple of patterns
($\{ p_1, \ldots, p_n \}$), or an alias to an existing pattern ($p = v$). Variables
names must be distinct in patterns, and all variable names are free.

\newcommand{\Pden}[2][\Gamma]{{P\llbracket{} #2 \rrbracket{}}_{#1}}
\begin{definition}[Pattern Types]
\begin{align*}
  \Pden{v
  } ={}& \Gamma(v)
  && \textsc{PVar} \\
  \Pden{l
  } ={}& \singleton{type(l)}{l}
  && \textsc{PLit} \\
  \Pden{ \{ p_1, \ldots, p_n\}
  } ={}& \{ \Pden{p_1}, \ldots, \Pden{p_n} \}
  && \textsc{PTuple} \\
  \Pden{p = v
  } ={}& \Pden{p} \land \Gamma(v)
  && \textsc{PAlias}
\end{align*}
\end{definition}

\subsection{Clauses}

A clause consists of a pattern paired with a guard.
\begin{align*}
  c ::= p\ \res{when}\ g
\end{align*}

\newcommand{\Cden}[2][\Gamma]{{C \llbracket{} #2 \rrbracket}_{#1}}

\begin{definition}[Clause Types]

  \begin{align*}
    \Cden{p\ \res{when}\ g} = \left\{ \Pden[\Gamma^{\prime}]{p}\ \middle|\ \Gamma^{\prime} \in \Gden[\Gamma]{g} \right\}
  \end{align*}
\end{definition}

\begin{definition}[Clause Types]
\end{definition}


\section{Type Atoms}


\begin{definition}[Type Atoms]
  A type atom is an indivisible type which is either positive
  ($\postype$), or the negation thereof ($\negtype$).

\begin{align*}
  \atomtype ::={}& \postype \alt \negtype \\
  \postype ::={}& \posstype \alt \posctype \\
  \posstype ::={}& \atom{any} \alt \atom{atom} \alt \atom{boolean} \alt
                   \atom{number} \alt \atom{integer} \\
  \posctype ::={}& \atom{tuple} \alt \{ {\postype}_{1}, \ldots, {\postype}_{n} \}
                   \alt \singleton{\posstype}{v} \\
  \negtype ::={}& \neg \postype
\end{align*}
\end{definition}

The Erlang datatypes of \atom{any}, \atom{atom}, etc.\ are positive simple types
(i.e.\ they are not compound, $\posstype$), while tuple types and singletons are
considered complex ($\posctype$). Our introduction of singleton types slightly
complicates the definition of type atoms over \citeauthor{Pearce2013}: we want
singletons to only contain simple positive types, so they must be defined
alongside the compound datatypes. This definition prevents tuples from
containing negative types, while also forbidding singletons containing
negative or compound types. Finally, there is no direct representation of the
\atom{none} type: the notation $\neg \atom{any}$ is used synonymously.

\subsection{Subtyping Relation}

\begin{mathpar}
  \inferrule* [right=S-Refl]
  { }
  { \postype \le \postype } % chktex 1

  \inferrule* [right=S-Any]
  { }
  { \postype \le \atom{any} } % chktex 1

  \inferrule* [right=S-Boolean]
  { }
  { \atom{boolean} \le \atom{atom} } % chktex 1

  \inferrule* [right=S-Integer]
  { }
  { \atom{integer} \le \atom{number} } \\ % chktex 1

  \inferrule* [right=S-Singleton1]
  { \posstypes \le \postype\\ % chktex 1
    v : \posstypes } % chktex 1 % chktex 26
  { \singleton{\posstypes}{v} \le \postype } % chktex 1

  \inferrule* [right=S-Singleton2]
  { \posstypes \le \posstype\\ % chktex 1
    v =:= w }
  { \singleton{\posstypes}{v} \le \singleton{\posstype}{w} } \\ % chktex 1
    
  \inferrule* [right=S-Tuple1]
  { }
  { \{ \postype[1], \ldots, \postype[n] \} \le \atom{tuple} } % chktex 1

  \inferrule* [right=S-Tuple2]
  { \postypes[1] \le \postype[n] \\ % chktex 1
    \cdots \\ % chktex 11
    \postypes[n] \le \postype[n] } % chktex 1
  { \{ \postypes[1] \ldots, \postypes[n] \} \le \{ \postype[1], \ldots, \postype[n] \} } % chktex 1
\end{mathpar}

We only define subtyping over positive type atoms, as per~\cite{Pearce2013}.
This relation is easy to define. All positive type atoms are subtypes of
themselves (\textsc{S-Refl}), and \atom{any} is at the top of the type
hierarcy (\textsc{S-Any}). Next, the axioms of Erlang's type system are encoded,
cementing the relation between simple positive type atoms (\textsc{S-Boolean},
\textsc{S-Integer}). Following up quickly behind, we allow these types to
be restricted to specific values via the instantiation of a singleton via
\textsc{S-Singleton1}. Singletons can be compared to oneanother via
\textsc{S-Singleton2}, which ensures that the sub-typing relation holds, and
that the values are equal via Erlang's $=:=$ operator. We prefer $=:=$ over
$==$ as the latter performs type conversion between integers and floats, which
is unwanted behaviour. The \atom{tuple} type can be restricted to a tuple of
a specific number of elements via \textsc{S-Tuple1}, that is to say a tuple with
$n$ elements is a subtype of a tuple with any number of elements. Finally, the
relation distributes across tuples, maintaining the size, order, and types of
elements (\textsc{S-Tuple2}).

\subsection{Intersection}

We take a gigantic shortcut here:

\begin{definition}[Positive Type Intersection]
  \newcommand{\intersect}{\sqcap}
  \begin{align}
    \postype \intersect \postype & = \postype \label{eqn:irefl} \\
    \postypes \intersect \atom{any} & = \postypes \label{eqn:iany} \\
    \atom{boolean} \intersect \atom{atom} & = \atom{boolean} \label{eqn:ibool} \\
    \atom{integer} \intersect \atom{number} & = \atom{integer} \label{eqn:iint} \\
    \left\{ {\postypes}_1, \ldots, {\postypes}_n \right\} \intersect \left\{ {\postype}_1, \ldots, {\postype}_m \right\}
                                   & = \begin{cases}
                                     \atom{none} & \text{if}\ n \ne m \\
                                     \atom{none} & \text{if}\ \exists{} i . {\postypes}_i \intersect {\postype}_i = \atom{none} \\
                                     \left\{ {\postypes}_1 \intersect {\postype}_1, \ldots, {\postypes}_n \intersect {\postype}_n \right\} & \text{otherwise}
                                   \end{cases} \label{eqn:ituple} \\
    \singleton{\postypes}{v} \intersect \singleton{\postype}{w}
                                    & = \begin{cases}
                                      \atom{none} & \text{if}\ v\ \res{=/=}\ w \\
                                      \atom{none} & \text{if}\ \postypes \intersect \postype = \atom{none} \\
                                      \singleton{\postypes \intersect \postype}{v} & \text{otherwise}
                                    \end{cases} \label{eqn:isingle1} \\
    \singleton{\postypes}{v} \intersect \postype
                                 & = \begin{cases}
                                   \atom{none} & \text{if}\ \postypes \intersect \postype = \atom{none} \\
                                   \singleton{\postypes \intersect \postype}{v} & \text{otherwise}
                                   \end{cases} \label{eqn:isingle2}                                                                                     
  \end{align}
\end{definition}
The rules are tried in order, from top to bottom. If no rule matches, the result
is $\atom{none}$. Rules~\ref{eqn:iany},~\ref{eqn:ibool},~\ref{eqn:iint}
and~\ref{eqn:isingle2} are symmetric.


\section{Normalisation and Canonicalisation}

\newcommand{\rewrite}{\Longrightarrow}

\subsection{Disjunctive Normal Form}

\begin{definition}[Disjunctive Normal Form]\label{def:dnf}
\begin{alignat}{2}
  & \neg \neg T && \rewrite T \\
  & \neg (S \lor T) && \rewrite (\neg S) \land (\neg T) \\
  & \neg (S \land T) && \rewrite (\neg S) \lor (\neg T) \\
  & (S \lor T) \land U && \rewrite (S \land U) \lor (T \land U) \\
  & \{ \ldots, S \lor T, \ldots \} && \rewrite \{ \ldots, S, \ldots \} \lor \{ \ldots, T, \ldots \} \\
  & \{ \ldots, S \land T, \ldots \} && \rewrite \{ \ldots, S, \ldots \} \land \{ \ldots, T, \ldots \} \\
  & \{\ldots \neg T, \ldots \} && \rewrite \{ \ldots, \atom{any} , \ldots \} \land \neg \{ \ldots, T, \ldots \}
\end{alignat}

$DNF(T) = T \rewrite^{*} T^{\prime}$ such that no more rewrite rules
apply~\cite{Pearce2013}.
\end{definition}

\subsection{Conjunct Canonicalisation}

\begin{definition}[Conjunct Canonicalisation]\label{def:can}
  \begin{alignat}{3}
    & \atom{none} \land T && \rewrite \atom{none} \\
    & \postypes \land \postype && \rewrite \postypes \sqcap \postype \\
    & \postypes \land (\neg \postype) && \rewrite \atom{none} && \quad \text{if}\ \postypes \le \postype \\
    &                              && \rewrite \postypes && \quad \text{if}\ \postypes \sqcap \postype = \atom{none} \\
    &                              && \rewrite \postypes \land \neg (\postypes \sqcap \postype) && \quad \text{if}\ \postype \nleq \postypes \\
    & (\neg \postypes) \land (\neg \postype) && \rewrite \neg \postypes && \quad \text{if}\ \postype \le \postypes
\end{alignat}

$CAN(T) = T \rewrite^{*} T^{\prime}$ such that no more rewrite rules
apply~\cite{Pearce2013}.
\end{definition}

\begin{definition}[Enhanced Disjunctive Normal Form]
\begin{align*}
  {DNF}^{+}(T) = CAN(DNF(T))
\end{align*}
\end{definition}

\section{A Complete Subtyping Relation}

\begin{align*}
\end{align*}

\section{Implementation}

\subsection{Exhaustive Rewriting}
  
We define a bottom-up mapping over types, where $F : T \rightarrow T$.

\begin{align*}
  map(F, \atom{any}) ={}& F(\atom{any}) \\
  map(F, \atom{atom}) ={}& F(\atom{atom}) \\
  map(F, \atom{boolean}) ={}& F(\atom{boolean}) \\
  map(F, \atom{number}) ={}& F(\atom{number}) \\
  map(F, \atom{integer}) ={}& F(\atom{integer}) \\
  map(F, \atom{tuple}) = {}& F(\atom{tuple}) \\
  map(F, \singleton{T}{v}) ={}& F(\singleton{map(F, T)}{v}) \\
  map(F, \{ T_1, \ldots, T_n \}) ={}& F( \{ map(F, T_1), \ldots, map(F, T_n) \}) \\
  map(F, \neg T) ={}& F(\neg map(F, T)) \\
  map(F, S \lor T) ={}& F(map(F, S) \lor map(F, T)) \\
  map(F, S \land T) ={}& F(map(F, S) \land map(F, T))
\end{align*}


Rewriting of terms for DNF (Definition~\ref{def:dnf}) and CAN
(Definition~\ref{def:can}) requires that rules be applied to the type until no
more apply. This can be achieved by recursively picking redexes of the type
and repeatedly applying a single-step rewriting function until the result does
not change. Naively, this can be achieved by repeatedly mapping the single-step
rewriting function over the type until the result of mapping is unchanged.

\begin{align*}
  rewrite(F, T) ={}&
                     \begin{cases}
                       T & \text{if}\ map(F, T) = T \\
                       rewrite(F, map(F, T)) & \text{otherwise}
                     \end{cases}
\end{align*}

Therefore $T \rewrite^{*} T^{\prime}$ such that no more rewrite rules apply is
equivalently $rewrite(\rewrite, T)$.

\todo[inline]{do we need to do this? should we not just map rewrite over the
  type? I don't think so. We're pushing things \textit{down} the tree as well
  as up, so I think we might need to re-visit already re-written redexes}

\section{Message Dependencies}

\subsection{Out-of-order dependencies}

It is possible for the reception of messages to rely on messages sent in the
\textit{future}:

\inputminted{erlang}{order.erl}

The client will send the messages \texttt{'end'} and \texttt{'begin'} to the
server --- \textit{in that order}. Due to Erlang's selective \texttt{receive}
behaviour, the server receives the messages in the opposite order:

\begin{minted}[autogobble]{text}
  Erlang/OTP 20 [erts-9.3.3.3] [source] [64-bit] [smp:8:8] [ds:8:8:10] [async-threads:10] [kernel-poll:false]

  Eshell V9.3.3.3  (abort with ^G)
  1> order:start().
  done!
  'begin'
  2>
\end{minted}

It seems that there is a general rule: if receiving $x$ relies on receiving $y$,
then $x$ \textit{could} be received if $y$ is \textit{ever} sent.

\subsection{Singleton Construction}

The construction of a singleton type $\singletonex$ requires that we know the
type of the Erlang term $v$. As Erlang's type system is dynamic, we construct
a function which determines the type of a term via guard evaluation. Note that if
$S \le T$, then the function clause for typing terms of $S$ must come before the
clause for $T$, as function clauses are considered in order:

\begin{minted}[autogobble]{erlang}
  type_of(X) when is_boolean(X) -> t_boolean();
  type_of(X) when is_atom(X)    -> t_atom();
  type_of(X) when is_integer(X) -> t_integer();
  type_of(X) when is_number(X)  -> t_number();
\end{minted}

This allows for easy construction of a singleton:
\begin{minted}[autogobble]{erlang}
  t_singleton(T) -> #t_singleton{type=type_of(T),value=T}.
\end{minted}


\bibliographystyle{ACM-Reference-Format}
\bibliography{intro}

\end{document}